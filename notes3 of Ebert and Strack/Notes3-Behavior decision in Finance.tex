\documentclass[11pt,aspectratio=169]{beamer}
\usepackage[T1]{fontenc}
\usepackage[utf8]{inputenc}
\usepackage{lmodern}
\usepackage{booktabs,tabularx}
\usepackage{graphicx}
\usepackage{amsmath, amssymb, amsfonts, amsthm, bbm}
\usepackage[
    natbib=true,
    bibencoding=inputenc,
    bibstyle=authoryear-ibid,
    citestyle=authoryear-comp,
    maxcitenames=3,
    maxbibnames=10,
    useprefix=false,
    sortcites=true,
    backend=bibtex
]{biblatex}
\AtBeginDocument{\toggletrue{blx@useprefix}}
\AtBeginBibliography{\togglefalse{blx@useprefix}}
\setlength{\bibitemsep}{1.5ex}
\addbibresource{References.bib}

\usepackage{hyperref}
\hypersetup{
    colorlinks=true,
    linkcolor=black,
    anchorcolor=black,
    citecolor=black,
    filecolor=black,
    menucolor=black,
    runcolor=black,
    urlcolor=black}

\DeclareMathOperator{\argmax}{arg\,max}

%Theorem
\newtheorem{assumption}{Assumption}
\newtheorem{proposition}{Proposition}
%\newtheorem{definition}{Definition}


\begin{document}
\mode<presentation>{	
    \setbeamertemplate{navigation symbols}{}
}

\title{Topics in Behavioral Decisons in Finance}
\date{Discussion by Christian Hilpert\\ \today}
\begin{frame}
    \titlepage
\end{frame}


\begin{frame}{4.3 Ebert and Strack, 2015, AER}
    \begin{itemize}
        \item dynamic perspective on gambling/ CPT decisions\medskip
        \item CPT agent never stops gambling/ investing\medskip
        \item "skewness preference in the small"\medskip
        \item $\Rightarrow$ simple, small, lottery-like risk with negative expectation.\medskip
        \item If I lose a bit more, I stop.\medskip
        \quad investment: never exercise attractive  options\medskip
        \quad key assumption: probability weighting domininate loss aversion\medskip
	\end{itemize}
\end{frame}



\begin{frame}{4.3 Ebert and Strack, 2015, AER}
    \begin{itemize}
        \item  Formally, binary risk:  $ L=(g,p,L,1-p) $\medskip
    \begin{equation}
        CPT = \begin{cases}
        [1-w^+(p)]V(L)+w^+(p)V(g),  & r \leq  L\\
        w^-(1-p)V(L)+w^+(p)V(g) ,  & L< r \leq g \\
        w^-(1-p)V(L)+[1-w^-(1-p)]V(g),  & g< r\\
        \end{cases}
    \end{equation}\medskip

    \item $w^-,w^+:[0,1] \rightarrow [0,1]$  , non-decreasing\medskip
    \item $w^+(0)=w^-(0)=0$\medskip
    \item $w^+(1)=w^-(1)=1$\medskip
    \item continuous, increasing V: $\mathbb{R} \rightarrow \mathbb{R} $ with $V(r)=0$\medskip
\end{itemize} 
\end{frame}

\begin{frame}{4.3 Ebert and Strack, 2015, AER}
    \textbf{Assumption 1}\\
    \begin{itemize}
        \item The value function U has finite left and right derivatives, 
       $\partial _-V(x)$ and $\partial _+V(x)$ at every wealth level $x$.\medskip
        \item Further, $\lambda = \sup _{x\in \mathbb{R} } \frac{\partial _-V(x)}{\partial _+V(x)} < \infty$ exists.\medskip
        \item $\Rightarrow$ ??? too extreme, no infinite loss aversion(KT,1979 ??)\medskip
    \end{itemize}
    \textbf{Assumption 2}\\
    There exists at least one $p \in (0,1)$ such that:
    \begin{enumerate}[(i)]
        \item $w^+(p)> \frac{\lambda}{1-p+\lambda p} =b_{\lambda}(p)$ \medskip
        \item $w^-(1-p)< \frac{1-p}{1-p+\lambda p} =b_{1/\lambda}(1-p) $  \medskip
	\end{enumerate}
(satisfied by any dominan function of $w$)\\
\end{frame}


\begin{frame}{4.3 Ebert and Strack, 2015, AER}
    \begin{itemize}
        \item $b_\theta (p)=\frac{\theta p}{1-p+\theta p} $ , $ (\theta=1)$ \medskip
        \item $b_\theta(0)=0, b_\theta(1)=1$\medskip
        \item $\theta>1 \Rightarrow$ concave, above $45^{ \circ}$\medskip
        \item $\theta<1 \Rightarrow$ convex, below $45^{ \circ}$\medskip
        \item $\Rightarrow$ A2 satisfied: $\omega ^{+}(5\%)> b_\lambda(5\%) $;\quad $ \omega ^{-}(95\%)> b_\frac{1}{\lambda}(95\%) $\medskip
        \item At least one probability sufficiently overweighted by $\omega^+ : \omega^+(p)>b_{\lambda}(p) \geq p$\medskip
        \item and the complimentary probability $1-p$ is ?? by $\omega^- :\omega^-(1-p) < b_{\frac{1}{\lambda}}(1-p) \leq 1-p$ .\medskip
	\end{itemize}
\end{frame}
\begin{frame}{4.3 Ebert and Strack, 2015, AER}
 
    \begin{itemize}
        \item Under Assumptions 1 and 2, for every wealth level there exists an attractive zero-mean binary lottery that is arbitrarily small. \medskip
	\end{itemize}

    \textbf{Corollary}\\
    \begin{itemize}
        \item  Under Assumptions 1 and 2, for every wealth level there exists an attractive, arbitrarily small binary lottery with negative mean.\medskip
	\end{itemize}
    \begin{itemize}
        \item generally, risk aversion is defined as every fair risk\medskip
        \item Here: skewness preference in the small thus implies that, at every wealth level, a CPT agent is not risk averse\medskip
        \item $\Rightarrow $  sufficiently small risks are attractive to CPT\medskip
        \item   Azevedo and Gottlieb (2012) $\Rightarrow$ skewness preference in the large, $p\approx 7.2$ for CPT weight $\Rightarrow$ not so small!\medskip
	\end{itemize}
\end{frame}

\begin{frame}{4.3 Ebert and Strack, 2015, AER}
    \textbf{Dynamic Consequences}\\
    \begin{itemize}
        \item what does skewness preference in the small do?\medskip
        \item consider  Markov diffusion $ X= (X_t)_{t \in \mathbb{R}_+ }$ ; $dX_t = \mu(X_t)dt + \sigma (X_t)dW_t $,\medskip
        \item $(W_t)_{t \in \mathbb{R}_+ }$: a Brownian motion\medskip
        \item $\mu:\mathbb{R} \rightarrow \mathbb{R} $ \medskip
        \item $\sigma : \mathbb{R}  \rightarrow (0,\infty) $ Lipschitz continuous\medskip
        \item Investment or gambling strategies : integrable stopping times $\tau $, adapted filtration $(\mathcal{F} _t)_{t \in \mathbb{R}_+ } $ (all avaliable information)\medskip
        \item CPT utility of strategy $\tau$ given $\mathcal{F} _t$, $CPT(X_\tau,\mathcal{F}_t)$ (time inconsistency)\medskip
        \item $\Rightarrow \tau$ may be changed later\medskip
       	\end{itemize}
\end{frame}


\begin{frame}{4.3 Ebert and Strack, 2015, AER}
    \begin{itemize}
        \item A naive investor does not anticipate that\medskip
        \item $\Rightarrow$ stops at $t$ if  $CPT(X_\tau,\mathcal{F}_t) \leq CPT(X_t,\mathcal{F}_t)=U(X_t) $\medskip
        \item $U(X_t) \geq \sup_{\tau \geq t} CPT(X_\tau,\mathcal{F}_t)$
    \end{itemize}
 \textbf{Theorem 2}\\
    \begin{itemize}
        \item Under Assumptions 1 and 2, the naive CPT agent never stops. (preference point can be dynamic and change)\medskip
        \item Information : simple two-threshold strategy: stops if utility drops, continue utility raised.\medskip
        \item a lot always exceeds stopping.\medskip
	\end{itemize}
\end{frame}


\end{document}
