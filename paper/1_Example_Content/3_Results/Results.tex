% !TeX program = pdflatex
% !BIB program = biber



\section{Results}
\label{sec:Results}

\blindtext
With this, we can test our hypotheses.

\subsection{Test of \autoref{hy:UNBAL}}

Our first result supports \autoref{hy:UNBAL}. \blindtext The analysis we conducted to obtain \autoref{r:main} is described in detail in \autoref{tmain:focus}. Let's reference a~section, a~subsection, and a~figure from the appendices: \autoref{sec:app:example_tables}, \autoref{sec:app:salience}, \autoref{fig:concentrated_cl}.

\begin{result} \label{r:main}
	\blindtext
\end{result}

\begin{table}[t]
	\caption{An Example Table}
	\label{tmain:focus}
	\begin{tabularx}{\textwidth}[t]{L T{1.3}}
		\toprule
		Dependent variable & {$\hat{d}$} \\
		\midrule
		Estimate      &       0.123{***} \\
		&     (0.011) \\
		\midrule
		Observations & {750} \\
		%clustered by subjects & \\
		Subjects      &         {250} \\
		\bottomrule
	\end{tabularx}
	\tablenotes{%
		Standard errors in parentheses, clustered on the subject level.
		{*} $p < 0.10$, {**} $p < 0.05$, {***} $p<0.01$.
	}
\end{table}

\blindtext

\subsection{Test of \autoref{hy:UNBAL_2}}
\blindtext We thereby test \autoref{hy:UNBAL_2}.

\begin{result} \label{r:main2}
	\blindtext
\end{result}

Our second result provides evidence in support of \autoref{hy:UNBAL_2}.\blindmathfalse\blindtext\blindmathtrue

\subsection{Heterogeneity}
\blindmathpaper
