% !TeX program = pdflatex
% !BIB program = biber


\section{Discussion}
\label{sec:Discussion}

\blindmathfalse
\subsection{Some Limitations}
\label{sec:discussion:inspection_costs}

Let's reference some tables: \autoref{tab:basic_formatting} and \autoref{tab:advanced_formatting}. \blindtext

% !TeX program = pdflatex
% !BIB program = biber



\begin{table}[t]
	\caption{Points Awarded in Our Typeface Competition---Basic Formatting}
	\label{tab:basic_formatting}
	\begin{tabular}{@{}lccccc@{}}
		\toprule
					&	Utopia	&	Computer Modern	&	Charter	&	Times Roman	&	Palatino	\\
		\midrule
		Yo\"{e}l	&	1	&	1	&	2	&	0	&	1	\\
		\c{C}elik	&	2	&	0	&	2	&	1	&	0	\\
		Anit\`{a}	&	1	&	2	&	1	&	2	&	0	\\
		U\u{g}ur	&	1	&	2	&	0	&	1	&	0	\\
		H\r{a}kan	&	1	&	0	&	2	&	0	&	1	\\
		Allison		&	2	&	0	&	1	&	2	&	1	\\
		P\'{i}a		&	1	&	0	&	2	&	1	&	0	\\
		David		&	1	&	0	&	2	&	1	&	1	\\
		\midrule
		Sum			&  10	&	5	&	12	&	8	&	4	\\
		\bottomrule
	\end{tabular}
\end{table}

\begin{table}[b]
	\caption{Points Awarded in Our Typeface Competition---More Sophisticated Formatting}
	\label{tab:advanced_formatting}
	\sisetup{table-column-width = 0.13\textwidth}
	% <{\qquad} @{\extracolsep{\fill}}
	\renewcommand\cellalign{ct}
	\begin{tabularx}{\textwidth}{ @{} L *{5}{T{2.0}} }
		\toprule
			&	{Utopia\parnote{\texttt{\textbackslash usepackage\{fourier\}}}}
			&	{\makecell[{{p{0.13\textwidth}}}]{\centering Computer Modern\parnote{The \LaTeX\ standard serif font.}}}
			&	{Charter\parnote{\texttt{\textbackslash usepackage[charter]\{mathdesign\}}}}
			&	{\makecell[{{p{0.13\textwidth}}}]{\centering Times Roman\parnote{\texttt{\textbackslash usepackage\{newtxtext, newtxmath\}}}}}
			&	{Palatino\parnote{\texttt{\textbackslash usepackage[sc]\{mathpazo\}}}}
			\\
		\midrule
		Yo\"{e}l  	& 	1	&	1	&	2	&	0	&	1	\\
		\c{C}elik	&	2	&	0	&	2	&	1	&	0	\\
		Anit\`{a}	&	1	&	2	&	1	&	2	&	0	\\
		U\u{g}ur	&	1	&	2	&	0	&	1	&	0	\\
		H\r{a}kan	&	1	&	0	&	2	&	0	&	1	\\
		Allison		&	2	&	0	&	1	&	2	&	1	\\
		P\'ia		&	1	&	0	&	2	&	1	&	0	\\
		David		&	1	&	0	&	2	&	1	&	1	\\
		\midrule
		Sum			&  10	&	5	&	12	&	8	&	4	\\
		\bottomrule
		%\multicolumn{6}{@{} p{\textwidth} @{}}{%
		%	\renewcommand{\parnotevskip}{-22pt}
		%	\parnotes%
		%	\vspace{\parnotevskip}
		%}
		% {\justify\setlength{\parindent}{0pt}\parnotes}
	\end{tabularx}
	%\renewcommand{\parnotevskip}{-22pt}
	\parnotes%
\end{table}


\subsection{Utility from Money}
\label{sec:discussion:utility_from_money}

In deriving our predictions (\autoref{sec:Predictions}), we assume that subjects base their decisions on utility derived from receiving monetary payments $c_t$ at various dates~$t$. We also make the standard assumption that utility from money is increasing in its argument but not convex, i.e., $u'(c_t) \ge 0$ and $u''(c_t) \le 0$. Both assumptions are frequently made in studies on intertemporal decision making.

One way to justify the assumption of utility being based on money---rather than consumption---is that individuals anticipate to consume the payments that they receive at date~$t$ within a~short period around $t$. Given that the maximum payment was below \euro 20 and that any two payment dates were separated by at least two weeks, this seems reasonable \citep[see the arguments in favor of this view in][]{Halevy2014}.

A~second justification is consistency within the discipline: \cite{Halevy2014} points out that ``in the domain of risk and uncertainty \dots\ preferences are often defined over payments.'' In line with this,
\cite[p.\,62]{Koszegi2013} make the same assumption of ``money in the utility function'':

\begin{quotation}
in some applications we also assume that monetary transactions induce \emph{direct} utility consequences, so that for instance an agent making a~payment experiences an~immediate utility loss. The idea that people experience monetary transactions as immediate utility is both intuitively compelling and supported in the literature: \dots\ some evidence on individuals' attitudes toward money, such as narrow bracketing (\dots) and laboratory evidence on hyperbolic discounting (\dots), is difficult to explain without it.
\end{quotation}

Last but not least, the papers by \cite{McClure2004a, McClure2007} demonstrate that brain activation, as measured by functional magnetic resonance imaging, is similar for primary and monetary rewards.

Let us now discuss the second assumption: that utility from money is nonconvex. We find that subjects allocate more money to the concentrated payoffs in the unbalanced than in the associated balanced budget sets---which we call concentration bias. One might argue that this relative preference for concentrated payoffs can be explained by the per-period utility function over money being convex.

Obtaining evidence on the shape of utility over money is nontrivial because it requires that at least two monetary amounts be compared with each other without the one clearly dominating the other. Thus, estimates of the curvature of the utility function over money can be obtained in two ways: the monetary amounts must be paid in different states of the world, i.e., comprise a~lottery, or they have to be paid at different points in time.\footnote{As a~matter of fact, the latter was the motivation behind \cite{Samuelson1937}: ``Under the following four assumptions, it is believed possible to arrive theoretically at a~precise measure of the marginal utility of \textit{money income} \dots'' (p.\,155; emphasis in the original).} Both methods entail particular theoretical assumptions.

\cite{Andersen2008} advocate the former approach and argue that when estimating time preference parameters, one should control for the curvature of the utility function through a~measure of the curvature that is based on observed choices under risk. Their study and numerous other studies on risk attitudes consistently reveal that the vast majority of subjects is risk-averse even over small stakes. Hence, for the vast majority of subjects, utility over money is concave according to this methodology (ruling out probability weighting). Others, most notably \cite{Andreoni2012}, have argued that the degree of curvature measured via risky choices probably overstates the degree of curvature effective in intertemporal choices, but they also find that utility is concave (albeit close to linear). Given this unambiguous evidence from previous studies, it is implausible that our subjects exhibit convex utility over money.
