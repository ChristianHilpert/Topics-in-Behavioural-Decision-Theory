% !TeX program = pdflatex
% !BIB program = biber



\renewcommand{\blindmarkup}[1]{\emph{#1}}
\blindmathfalse


Ample empirical evidence has challenged the paradigm of neoclassical economics that people are rational.
Cumulative Prospect Theory, developed in the seminal work of \citet{KahnemanTversky1979} and  \citet{TverskyKahneman1992}, has become the most successful descriptive theory of decisions under risk to date.
The aim of this lecture is twofold: First, we are going to discuss Cumulative Prospect Theory and its limitations as a descriptive theory of choice under risk.
Second, we are considering current alternatives to Cumulative Prospect Theory.
Our emphasis in this second part lies on two critical aspects of Cumulative Prospect Theory: First, it assumes that people judge outcomes of gambles against an exogenous reference point.
Second, the strength of its probability weighting of extreme probabilities implies implausible behaviour in the form of excessive skewness preference, that is, an excessive tendency to gambling.


\nocite{EbertStrack2015,EbertStrack2018Unpublished, Barberis2012a}\nocite{KoeszegiRabin2006,KoeszegiRabin2007,Shefrin2008,Starmer2000,Wakker2010}
\nocite{BordaloGennaioliShleifer2012,BordaloGennaioliShleifer2013a,BordaloGennaioliShleifer2013b}
\nocite{Rabin1998, Rabin2000, Thaler2016, Barberis2013a,Barberis2013b}
\nocite{AzevedoGottlieb2012}
